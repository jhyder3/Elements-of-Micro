\documentclass[compress]{beamer}
%\documentclass[handout]{beamer}

\mode<presentation>
{
  \usetheme{CambridgeUS}      % or try Darmstadt, Madrid, Warsaw, ...
  \usecolortheme{default} % or try albatross, beaver, crane, ...
  \usefonttheme{default}  % or try serif, structurebold, ...
  \setbeamertemplate{navigation symbols}{}
  \mode<beamer>{\setbeamertemplate{blocks}[rounded][shadow=true]}
  \setbeamertemplate{caption}[numbered]
  \useoutertheme{infolines}
  \useoutertheme[subsection=false]{miniframes}
} 

\usepackage[english]{babel}
\usepackage[utf8x]{inputenc}
\usepackage{pifont}
\usepackage{amssymb}
\usepackage{xcolor}
\usepackage{tikz}
\newcommand{\xmark}{\ding{55}}%
\usepackage{eurosym}
\usepackage{graphicx}
\usepackage{comment}
% set colors
\definecolor{myNewColorA}{RGB}{0, 45,114}
\definecolor{myNewColorB}{RGB}{0, 45,114}
\definecolor{myNewColorC}{RGB}{0, 45,114} % {130,138,143}
\setbeamercolor*{palette primary}{bg=myNewColorC}
\setbeamercolor*{palette secondary}{bg=myNewColorB, fg = white}
\setbeamercolor*{palette tertiary}{bg=myNewColorA, fg = white}
\setbeamercolor*{titlelike}{fg=myNewColorA}
\setbeamercolor*{title}{bg=myNewColorA, fg = white}
\setbeamercolor*{item}{fg=myNewColorA}
\setbeamercolor*{caption name}{fg=myNewColorA}
\setbeamercolor{date in head/foot}{fg=white}
\setbeamercolor{page number in head/foot}{fg=white}


\titlegraphic{%
\vspace{0cm}
    \includegraphics[width=4cm]{logo-vertical2.png}
}

\usepackage{subcaption}
% \usepackage{mathrsfs}

\usepackage{dirtytalk}
\usepackage{tcolorbox}
\usepackage{multicol}
\usepackage{multirow}
\usepackage{caption}
\usepackage{threeparttable}
\usepackage{pdfpages}
\usepackage{longtable}
\usepackage{adjustbox}
\usepackage{colortbl}
\usepackage{tikz}
\def\checkmark{\tikz\fill[scale=0.4](0,.35) -- (.25,0) -- (1,.7) -- (.25,.15) -- cycle;} 
\usepackage{pdfpages}

\usepackage{accents}
\newcommand{\ubar}[1]{\underaccent{\bar}{#1}}
%\usepackage{enumitem}

%% References - BEGIN
\usepackage[backend=bibtex8,style=authoryear-icomp,doi=false,url=false,isbn=false,eprint=false]{biblatex}
\renewbibmacro{in:}{}		% gets rid of the 'In' in front of the journal name
%\bibliographystyle{natbib}        
% \bibliography{references.bib}

% diagram (tree)
\usepackage{tikz}
\tikzset{
  treenode/.style = {shape=rectangle, rounded corners,
                     draw, align=center,
                     top color=white,
                     bottom color=blue!20},
  root/.style     = {treenode, font=\Large,
                     bottom color=red!30},
  env/.style      = {treenode, font=\ttfamily\normalsize},
  dummy/.style    = {circle,draw}
}

% Title page details: 
\title{Chapter 6: Supply, Demand, and Government Policies} 
\author{Jamie Hyder \\
    Discussion section 4}
\date{September 26, 2023}

\begin{document}

% Title page
\begin{frame}
    \titlepage 
\end{frame}

\begin{frame}{Outline}
    Today we will talk about government policies and their unintended consequences:
    \begin{enumerate}
        \item Price controls
        \item Social Welfare
    \end{enumerate}
\end{frame}

\begin{frame}{Price Controls}
\begin{block}{Two kinds of price controls:}
    \begin{itemize}
        \item Price \textit{Ceiling}: a legal \textbf{maximum} on the price at which a good can be sold
        \item Price \textit{Floor}: a legal \textbf{minimum} on the price at which a good can be sold
    \end{itemize}
\end{block}

\begin{block}{Price controls can be:}
    \begin{itemize}
        \item \textit{Binding}: market forces push price towards equilibrium, but due to the price control it is unable by law to reach equilibrium
        \item \textit{Non Binding:} market forces push price towards equilibrium, and the price controls do not prevent reaching equilibrium
    \end{itemize}
\end{block}
\end{frame}

\begin{frame}{Price Controls}
    Let's consider the market for coffee: In equilibrium
    \begin{itemize}
        \item $Q^* = 100$
        \item $P^* = 3$
    \end{itemize}
\end{frame}

\begin{frame}{Price controls}
Let's consider 2 cases of a price \textit{ceiling}:
\begin{itemize}
\item $P = 2.5$
\item $P = 3.5$
\end{itemize}

\begin{block}{In either case}
    \begin{itemize}
        \item What will $Q_D$ and $Q_S$ be?
        \item Does the ceiling cause a shortage or surplus?
        \item Is the policy binding?
    \end{itemize}
\end{block}
\end{frame}


\begin{frame}{Price Controls}
Let's consider 2 cases of a price \textit{floor}:
\begin{itemize}
\item $P = 2.5$
\item $P = 3.5$
\end{itemize}

\begin{block}{In either case}
    \begin{itemize}
        \item What will $Q_D$ and $Q_S$ be?
        \item Does the ceiling cause a shortage or surplus?
        \item Is the policy binding?
    \end{itemize}
\end{block}
\end{frame}

\begin{comment}
\begin{frame}{Taxes}
    In the U.S. (Not necessarily in other countries!) direct price controls are relatively rare.

\medskip

Taxes are much more common: \textbf{What is a tax?}

\medskip

\begin{block}{When taxes are levied we are concerned with:}
    The \textbf{Incidence of the tax}: how the burden of the tax is shared among participants (buyers \& sellers) in a market
\end{block}
\end{frame}

\begin{frame}{Taxes}
    Let's return to the market for coffee...

    \begin{block}{Consider a $\$0.5$ tax on each cup of coffee levied on the sellers:}
        \begin{itemize}
            \item Does this cause a shift in supply or demand?
            \item What will happen to $P^*$ or $Q^*$?
            \item What portion falls on buyers and what portion falls on sellers?
        \end{itemize}
    \end{block}
\end{frame}

\begin{frame}{Taxes}
    In this example, buyers and sellers share the incidence of the tax.

    \begin{block}{Let's consider the same questions under the following extreme cases:}
\begin{itemize}
    \item Demand is perfectly inelastic
    \item Supply is perfectly inelastic
\end{itemize}
    \end{block}
\end{frame}

\begin{frame}{Taxes \& Elasticity}
\begin{itemize}
    \item \textbf{Inelastic demand}: consumers buy the same amount of coffee no matter what $\implies$ consumers bear the entire incidence of the tax
    \item \textbf{Inelastic supply}: suppliers provide the same amount of coffee no matter what $\implies$ suppliers bear the entire incidence of the tax
\end{itemize}

\end{frame}


\begin{frame}{Taxes \& Elasticity}
   \begin{block}{ The relative elasticities determine the Tax Wedge: }
       This tells us which side of the market pays the what portion of the tax.

       \medskip

       In general:
       \begin{itemize}
           \item Demand is more elastic than supply $\to$ \textit{suppliers} bear more of the burden
           \item Supply is more elastic than demand $\to$ \textit{buyers} bear more of the burden
       \end{itemize}
   \end{block}

   Really it is the \textbf{relative} elasticities which determine the wedge.
\end{frame}
\end{comment}

\begin{frame}{Social Welfare}
Some key vocabulary:
\begin{itemize}
    \item Willingness to pay: max price a buyer will pay for a good
    \item Cost: value of everything a seller gives up to produce a good
\end{itemize}

\begin{block}{We have two notions of welfare surplus:}
    \begin{itemize}
        \item \textbf{Consumer surplus (CS):} amount a buyer is willing to pay minus the actual price they pay
        \item \textbf{Producer surplus (PS): } amount a seller is paid for a good minus the sellers cost of producing it
    \end{itemize}
\end{block}

    \medskip

    What does this look like on a supply and demand graph?
\end{frame}

\begin{frame}{Social Welfare}
\begin{block}{   We can also calculate the total surplus (TS) in society:}
    \begin{align*}
        \text{TS} &= (\text{value to buyers - price}) + (\text{price - cost to sellers})\\
        &= \text{value to buyers} - \text{cost to sellers}
    \end{align*}
\end{block}

\vspace{3mm}

\textit{We say that a resource allocation is \textbf{efficient} if it maximizes total surplus.}
\end{frame}

\begin{frame}{Social Welfare}
    \begin{block}{When there is a market distortion, the total surplus decreases:}
        We call this Dead Weight Loss (DWL): the fall in total surplus resulting from a market distortion (like a price control or a tax)
    \end{block}

    \begin{block}{The DWL from a market distortion will depend on the relative elasticities of supply and demand:}
        \begin{itemize}
            \item Inelastic supply/demand $\implies$ small DWL
            \item Elastic supply/demand $\implies$ large DWL
        \end{itemize}
    \end{block}
    \begin{center}
\textit{       ** The larger the distortion, the larger the DWL **
}    \end{center}
\end{frame}

\begin{frame}{DWL Example}
    \begin{block}{Let's consider the \textit{market for labor}...}
    If we impose a binding \textbf{minimum wage law}, what happens to:
    \begin{itemize}
        \item producer surplus
        \item consumer surplus
        \item dead weight loss
    \end{itemize}
    \end{block}

    \begin{block}{Now let's consider the \textit{market for NYC apartments}..}
    If we impose a binding \textbf{rent control}, what happens to:
    \begin{itemize}
        \item producer surplus
        \item consumer surplus
        \item dead weight loss
    \end{itemize}
    \end{block}
\end{frame}


\end{document}