\documentclass[12pt]{article}

%%%%%%%%%%%%%%%%%%%%%%%%%%%%
%%%%%%%%%%%%%%%%%%%%%%%%%%%%
% Load in packages
\usepackage{amsmath}
\usepackage{float}
\usepackage{amssymb}
\usepackage{hyperref}
\usepackage{graphicx}
\usepackage{enumitem}
\usepackage[top=1in, bottom=1in, left=1in, right=1in]{geometry}

%%%%%%%%%%%%%%%%%%%%%%%%%%%%
%%%%%%%%%%%%%%%%%%%%%%%%%%%%

\begin{document}

\begin{center}
\Large Chapter 21 \& 13 Practice Problems

\medskip

\normalsize Elements of Microeconomics (discussion section 4)

\medskip

\small Jamie Hyder
\end{center}

\section*{Question 1}
A consumer would like to purchase pasta and wine. They have \$100 to spend, pasta costs \$20 per plate and wine costs \$10 per glass.
\begin{enumerate}
    \item How much of each good could they possibly consume?
    \item What is the equation for their budget constraint?
    \item What does their budget constraint look like graphically?
    \item What happens if the price of pasta decreases to \$10?
    \item What happens if instead, income increases to \$160?
\end{enumerate}

\section*{Question 2}
Using the set up from question 1:
\begin{enumerate}
    \item Draw the indifference curve which represents the maximum satisfaction the consumer could possibly have given their initial budget constraint.
    \item Now consider the increase in income to \$160. Draw the new indifference curve which represents the maximum satisfaction the consumer could possibly have given this budget constraint.
    \item Now consider the decrease in price of pasta to \$10. Draw the new indifference curve which represents the maximum satisfaction the consumer could possibly have given this budget constraint AND illustrate the income and substitution effects as a result of the change. 
\end{enumerate}



\section*{Question 3}
     Consider a firm which produces pizzas. The following table shows the number of pizzas produced given varying numbers of workers:
    
\begin{table}[H]
        \centering
        \begin{tabular}{cc}
            Workers & Output \\
            \hline
            0 & 0 \\
            1 & 20 \\
            2 & 45 \\
            3 & 80 \\
            4 & 100 \\
            5 & 110 \\
        \end{tabular}

        \label{tab:placeholder}
    \end{table}
    
     A worker costs \$100 and the firm has fixed costs of \$200. Add columns for the marginal product, total cost, average total cost, and marginal cost and fill them in.
    


\end{document}
