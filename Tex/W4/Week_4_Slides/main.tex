\documentclass[compress]{beamer}
%\documentclass[handout]{beamer}

\mode<presentation>
{
  \usetheme{CambridgeUS}      % or try Darmstadt, Madrid, Warsaw, ...
  \usecolortheme{default} % or try albatross, beaver, crane, ...
  \usefonttheme{default}  % or try serif, structurebold, ...
  \setbeamertemplate{navigation symbols}{}
  \mode<beamer>{\setbeamertemplate{blocks}[rounded][shadow=true]}
  \setbeamertemplate{caption}[numbered]
  \useoutertheme{infolines}
  \useoutertheme[subsection=false]{miniframes}
} 

\usepackage[english]{babel}
\usepackage[utf8x]{inputenc}
\usepackage{pifont}
\usepackage{amssymb}
\usepackage{xcolor}
\usepackage{tikz}
\newcommand{\xmark}{\ding{55}}%
\usepackage{eurosym}
\usepackage{graphicx}
% set colors
\definecolor{myNewColorA}{RGB}{0, 45,114}
\definecolor{myNewColorB}{RGB}{0, 45,114}
\definecolor{myNewColorC}{RGB}{0, 45,114} % {130,138,143}
\setbeamercolor*{palette primary}{bg=myNewColorC}
\setbeamercolor*{palette secondary}{bg=myNewColorB, fg = white}
\setbeamercolor*{palette tertiary}{bg=myNewColorA, fg = white}
\setbeamercolor*{titlelike}{fg=myNewColorA}
\setbeamercolor*{title}{bg=myNewColorA, fg = white}
\setbeamercolor*{item}{fg=myNewColorA}
\setbeamercolor*{caption name}{fg=myNewColorA}
\setbeamercolor{date in head/foot}{fg=white}
\setbeamercolor{page number in head/foot}{fg=white}


\titlegraphic{%
\vspace{0cm}
    \includegraphics[width=4cm]{logo-vertical2.png}
}

\usepackage{subcaption}
% \usepackage{mathrsfs}

\usepackage{dirtytalk}
\usepackage{tcolorbox}
\usepackage{multicol}
\usepackage{multirow}
\usepackage{caption}
\usepackage{threeparttable}
\usepackage{pdfpages}
\usepackage{longtable}
\usepackage{adjustbox}
\usepackage{colortbl}
\usepackage{tikz}
\def\checkmark{\tikz\fill[scale=0.4](0,.35) -- (.25,0) -- (1,.7) -- (.25,.15) -- cycle;} 
\usepackage{pdfpages}

\usepackage{accents}
\newcommand{\ubar}[1]{\underaccent{\bar}{#1}}
%\usepackage{enumitem}

%% References - BEGIN
\usepackage[backend=bibtex8,style=authoryear-icomp,doi=false,url=false,isbn=false,eprint=false]{biblatex}
\renewbibmacro{in:}{}		% gets rid of the 'In' in front of the journal name
%\bibliographystyle{natbib}        
% \bibliography{references.bib}

% diagram (tree)
\usepackage{tikz}
\tikzset{
  treenode/.style = {shape=rectangle, rounded corners,
                     draw, align=center,
                     top color=white,
                     bottom color=blue!20},
  root/.style     = {treenode, font=\Large,
                     bottom color=red!30},
  env/.style      = {treenode, font=\ttfamily\normalsize},
  dummy/.style    = {circle,draw}
}

% Title page details: 
\title{Chapter 5: Elasticity and applications} 
\author{Jamie Hyder \\
    Discussion section 4}
\date{September 19, 2023}

\begin{document}

% Title page
\begin{frame}
    \titlepage 
\end{frame}

\begin{frame}{Outline}
    
    \begin{block}{Elasticity is an intuitive concept: }
        How do consumers change their behavior in response to changing prices or income?
    \end{block}

    \medskip

    We have different notions of elasticity, namely \textit{price elasticity} and \textit{income elasticity}... we will start by discussing price elasticity.
\end{frame}

\begin{frame}{Price elasticity}
    \begin{block}{We can consider price elasticity of \textit{supply} and \textit{demand}}
        \begin{itemize}
            \item \textbf{Price elasticity of demand}: how much $Q_D$ for a good responds to a change in the price of that good.
            \item \textbf{Price elasticity of supply}: how much $Q_S$ for a good responds to a change in the price of that good.
        \end{itemize}
    \end{block}
\end{frame}

\begin{frame}{Price Elasticity}
    A consumer or seller may be price \textbf{elastic} or \textbf{inelastic} for a particular good:
    \begin{block}{Elastic:}
        Quantity demanded/supplied responds \textit{a lot} to changes in price. \\
        \medskip
\textit{        Example: Travel
}    \end{block}
    \begin{block}{Inelastic:}
        Quantity demanded/supplied respond \textit{little} to changes in price. \\
        \medskip
\textit{        Example: Insulin
}    \end{block}
\end{frame}

\begin{frame}{Elasticity Influences}
    What factors will influence a good's elasticity?

    \begin{enumerate}
        \item \textbf{Availability of close substitutes}: other kinds of trucks, cars, bikes, etc.
        \item \textbf{Necessities vs. luxuries}: do you need it for work? For fun?
        \item \textbf{Market definition}: Are we considering the market for Ford F150s? For pickup trucks? For motor vehicles?)
        \item \textbf{Time horizon}: In the short run, maybe we need a pickup; in the long-run, maybe we retool our lives to accomadate a different car or no car at all
    \end{enumerate}
\end{frame}


\begin{frame}{Calculating elasticity}
    We have a simple equation to find an elasticity:
    \medskip
    \[X\; \text{Elasticity of}\; Y = | \frac{\%\Delta X}{\% \Delta Y}|\]
    
\vspace{3mm}

The price elasticity of demand is calculated as follows:
    $$
    \text{\textit{Price} elasticity of demand} = |\frac{\% \Delta Q_D}{\% \Delta P}|
    $$

\end{frame}

 
 \begin{frame}{Calculating Elasticity}
    \begin{block}{First, we need to know how to calculate the percent change of a price or quantity:}

    If good A used to cost \$10, and now it costs \$14, what is the percentage change?
    $$
    \dfrac{\text{Change in price}}{\text{Original price}} * 100\% = \dfrac{\$14 - \$10}{\$10} * 100\% = 40\%
    $$
    \end{block}
\medskip
    In our elasticity formula, we do not need to worry about multiplying by 100\%.
\end{frame}

\begin{frame}{Price elasticity of demand}
    Consider two points along a demand curve:
    \begin{itemize}
        \item A: price is \(P_A = 12\) and quantity demanded is \(Q_A = 60\)
        \item B: \(P_B = 8\) and \(Q_B = 80\)
    \end{itemize}
    \medskip
    We can use our formula to calculate the price elasticity of demand of:
    \begin{enumerate}
        \item Moving from A to B
        \item Moving from B to A
    \end{enumerate}
\end{frame}

\begin{frame}{Calculating price elasticity of demand}
\begin{enumerate}
    \item Moving from A to B: \(P_e = |\frac{\frac{80-60}{60}}{\frac{8-12}{12}} |= |\frac{\frac{1}{3}}{-\frac{1}{3}}| = |-1| = 1\)
    \item Moving from B to A: \(P_e = |\frac{\frac{60-80}{80}}{\frac{12-8}{8}}| = |\frac{-\frac{1}{4}}{\frac{1}{2}}| = |-\frac{1}{2}| = \frac{1}{2}\)
\end{enumerate}
\begin{center}
    ** We get two different price elasticities! What gives?? **
\end{center}

    
\end{frame}


\begin{frame}{The midpoint method}
    To avoid problems caused by calculating elasticities using different bases (as we saw in the previous example), we can use the midpoint method.

\medskip

\begin{block}{The midpoint method:}
Use the average of the two  points as the base in percentage calculations:

\[ \text{Price elasticity of demand} = \frac{\tfrac{Q_2 - Q_1}{(Q_2 + Q_1)/2}}{\tfrac{P_2 - P_1}{(P_2 + P_1)/2}} \]
\end{block}

\medskip

This is the formula we will use in this class!
\end{frame}
\begin{frame}{The midpoint method}
   \begin{block}{ Using the midpoint method with our previous example:}
    \begin{itemize}
        \item $P_A = 12$ and $Q_A = 60$
        \item $P_B = 8$ and $Q_B = 80$
    \end{itemize}
    \end{block}

    \begin{enumerate}
        \item What is the new base price?
            \begin{itemize}
                \item<2-> \(\frac{P_A + P_B}{2} = \frac{12+8}{2} = \$10\)
            \end{itemize}

        \item What is the new base quantity?
            \begin{itemize}
                \item<3-> \(\frac{Q_A + Q_B}{2} = \frac{60+80}{2} = 70\)
            \end{itemize}

        \item What is the percent change for quantity demanded?
            \begin{itemize}
                \item<4-> \(\frac{80-60}{70} = \frac{2}{7}\)
            \end{itemize}

        \item What is the percent change for price?
            \begin{itemize}
                \item<5-> \(\frac{12-8}{10} = \frac{2}{5}\)
            \end{itemize}
    \end{enumerate}

    \only<6->{%
    \begin{center}
        \textbf{Whether we move from A to B or B to A we get \(P_e = \tfrac{2/7}{2/5} = \tfrac{5}{7}\)}
    \end{center}}
\end{frame}

    \begin{frame}{The point method}
        We can also calculate the elasticity of demand at a particular point on the demand curve:
        \[\text{Price elasticity of Demand} = |\frac{\Delta Q_D}{\Delta P}| \times \frac{P}{Q_D}\]
        \begin{itemize}
            \item $\frac{\Delta Q_D}{\Delta P}$ is the reciprocal slope of the demand curve at the point $(Q_D,P)$
            \begin{itemize}
                \item (The slope of the demand curve is $\frac{\Delta P}{\Delta Q_D}$)
            \end{itemize}
        \end{itemize}
\begin{center}
    Let's do an example...
\end{center}

    \end{frame}

    \begin{frame}{The point method}
        Imagine we are given the following demand equation and point on the curve:
        \begin{itemize}
            \item Demand equation: \(P = -\frac{1}{2} Q + 10\)
            \item Point A: $Q_D = 80 \quad P = 8$ 
        \end{itemize}


\begin{block}{What is the price elasticity of demand at point A on the demand curve?}
        \begin{aligned}
            \text{Price elasticity of demand} &= \frac{1}{|\text{slope}|} \times \frac{P}{Q_D} \\
            & = \frac{1}{|-\frac{1}{2}|} \times \frac{8}{80} \\
            & = |-2| \times \frac{1}{10} \\
            & = |-\frac{1}{5}| \\
            & = \frac{1}{5}
        \end{aligned}
    
\end{block}
        
    \end{frame}


\begin{frame}{Cases of elasticity of demand}
Demand might be:
    \begin{itemize}
     \item \textbf{Elastic}: price change of $X\%$ $\to$ demand change greater than $X\%$ 
     \begin{itemize}
         \item Elasticity $>$ 1
     \end{itemize}
     \item \textbf{Inelastic}: price change of $X\%$ $\to$ demand change less than $X\%$ 
     \begin{itemize}
         \item Elasticity $<$ 1
     \end{itemize}
     \item \textbf{Unit elastic:} price change of $X\%$ $\to$ demand change of $X\%$
     \begin{itemize}
         \item Elasticity = 1
     \end{itemize}
     \item \textbf{Perfectly inelastic}: price change has no impact on demand
     \begin{itemize}
         \item Elasticity = 0
     \end{itemize}
     \item \textbf{Perfectly elastic:} small price change has enormous (infinite!) impact on demand
     \begin{itemize}
         \item This one is tricky...
     \end{itemize}
    \end{itemize}

\medskip
    \begin{center}
        Let's draw them!
    \end{center}
\end{frame}

\begin{frame}{Revenue}
The total revenue of a firm depends on the price and quantity sold of their products:

\[\text{Total revenue} = P \times Q\]

So, when a firm increases or decreases their prices $P$, the associated change in quantity sold $Q$ will determine their change in total revenue. \\

\medskip

What does total revenue look like on a graph?
    
\end{frame}

\begin{frame}{Revenue}
In order to determine the change in $Q$ caused by the change in $P$, we must use the price elasticity of demand...
\begin{block}{When demand is inelastic:}
    If price increases, total revenue increases
\end{block}    
\begin{block}{When demand is elastic:}
    If price increases, total revenue decreases
\end{block}
\begin{block}{When demand is unit elastic:}
    Total revenue remains constant when price changes
\end{block}
\end{frame}

\begin{frame}{Revenue}
    Let's say the price of a coffee at Starbucks initially was $P_A$, but has just doubled to $P_B = 2\times P_A$.

    \begin{block}{How will total revenue change when:}
    \begin{enumerate}
        \item Demand is elastic: quantity decreases by 75\%
        \item Demand is inelastic: quantity decreases by 25\%
        \item Demand is unit elastic
    \end{enumerate}
        
    \end{block}
\end{frame}

\begin{frame}{Different Elasticities}
    We have focused on the \textit{Price elasticity of demand}, but there are others.

\medskip
Income elasticity of demand:
    \begin{itemize}
        \item Positive for normal goods, negative for inferior goods
        \item $\text{income elasticity of demand} = |\dfrac{\%~\Delta Q_D}{\%~\Delta~\text{Income}}|$
    \end{itemize}

    \medskip
    \medskip

    Cross-price elasticity of demand:
    \begin{itemize}
        \item Positive for substitutes, negative for complements
        \item $\text{CP elasticity of demand} = |\dfrac{\%~\Delta Q_{D1}}{\%~\Delta P_2}|$
    \end{itemize}
    
\end{frame}


\begin{frame}{Price elasticity of supply}


\begin{block}{We use a very similar formula:}
\[    \text{Price elasticity of supply} = |\dfrac{\%~\Delta Q_S}{\%~\Delta P}|
\]
\end{block}

\begin{block}{Firms may have supply that is:}
    \begin{itemize}
        \item \textbf{Elastic}: an X\% change in price $\to$ $>X\%$ change in supply
        \begin{itemize}
            \item Elasticity $>$ 1
        \end{itemize}
        \item \textbf{Inelastic: }an X\% change in price $\to$ $<X\%$ change in supply
        \begin{itemize}
            \item Elasticity $<$ 1
        \end{itemize}
        \item \textbf{Unit elastic:} an X\% change in price $\to$ $X\%$ change in supply
        \begin{itemize}
            \item Elasticity = 1
        \end{itemize}
        \item \textbf{Perfectly inelastic}: any change in price $\to$ no change in supply 
        \begin{itemize}
            \item Elasticity = 0
        \end{itemize}
                \item \textbf{Perfectly elastic}: any change in price $\to$ enormous change in supply
        \begin{itemize}
            \item This one is tricky...
        \end{itemize}
    \end{itemize}
\end{block}
    
\end{frame}

\end{document}