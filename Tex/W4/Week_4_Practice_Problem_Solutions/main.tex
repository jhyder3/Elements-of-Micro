\documentclass[12pt]{article}

%%%%%%%%%%%%%%%%%%%%%%%%%%%%
%%%%%%%%%%%%%%%%%%%%%%%%%%%%
% Load in packages
\usepackage{amsmath}
\usepackage{float}
\usepackage{amssymb}
\usepackage{hyperref}
\usepackage{graphicx}
\usepackage[top=1in, bottom=1in, left=1in, right=1in]{geometry}

%%%%%%%%%%%%%%%%%%%%%%%%%%%%
%%%%%%%%%%%%%%%%%%%%%%%%%%%%

\begin{document}

\begin{center}
\Large Chapter 5 Practice Problem Solutions

\medskip

\normalsize Elements of Microeconomics (discussion section 4)

\medskip

\small Jamie Hyder
\end{center}

\medskip
\section*{Question 1}
\begin{enumerate}
\item Based on your intuition, choose 3 goods for which you think:
\begin{enumerate}
    \item Demand is inelastic
    \item Demand is elastic
    \item Supply is inelastic
    \item Supply is elastic
\end{enumerate}
It might be helpful to add a bit of justification. Does it matter what \textit{time frame} you're thinking about? Does the \textit{scope of the market} matter? Any other factors?

\vspace{5mm}

\textbf{Answer:}

\vspace{2mm}

There are many possible examples for each. Some ideas are:
\begin{enumerate}
    \item \textbf{Demand is inelastic}: Cigarettes, coffins, shoes, lightbulbs... anything for which people generally cannot do without. For all of these, the scope matters. The demand for shoes is inelastic because people need to wear shoes, but the demand for fancy basketball shoes might be fairly elastic, because people can always buy some cheaper sneakers shoes from walmart.
    \item \textbf{Demand is elastic}: 5-star hotels, flights to Cancun, kitchen renovations... the common theme is that these are all \textit{luxuries} for which people can do without. This might mean the \textit{time frame} matters: in the short-run a kitchen renovation is easy to put off, but in the long-run (over the course of, say, 50 years) demand may be fairly inelastic.
    \item \textbf{Supply is inelastic}: Stadium seats, houses, beef, microchips, missiles... these are all goods for which increasing supply \textit{in the short-run} is difficult or impossible.
    \item \textbf{Supply is elastic}: Gasoline, cigarettes, 2X4s... any good for which the inputs are cheap, storage is easy, and production can be ramped up or down quickly is likely to be elastic.
\end{enumerate}


\vspace{5mm}

\item Think about the market for Ford F150s. Do you expect demand to be elastic or inelastic? What about supply? Does this depend on any qualifiers about the time frame and the scope of the market?

\vspace{5mm}

\textbf{Answer:}

\vspace{2mm}

If we are thinking just about the \textit{market for F-150s}, we would expect demand to be fairly elastic:
\begin{itemize}
    \item People can choose a different type of pickup truck
    \item People can choose a non-truck vehicle
    \item People can choose to walk, bike, or take public transportation
\end{itemize}

Of course, this depends on the \textit{population of consumers} we are considering, which is part of our market definition. If we are considering the market for Ford F-150s for contractors, the demand may be less elastic, because they need these vehicles for work.

\vspace{2mm}

The time horizon also matters. In the short-run people might be able to make repairs on an existing vehicle, or go without a truck for some period of time. In the long-run, they may not be able to do without a new F-150.

\vspace{2mm}

There is much more discussion you could add here. Depending on your market definition (time-frame, scope of market, population of consumers) you may be able to argue that demand is inelastic \textit{or} elastic. The key is to back up your answer with sufficient economic reasoning.


\end{enumerate}

\section*{Question 2}
\begin{enumerate}
\item Take two points on a demand curve:
\begin{itemize}
    \item $P_A=12$ and $Q_A=60$
    \item $P_B=8$ and $Q_B=80$
\end{itemize}

Moving from A to B, what is the price elasticity of demand? Show each step clearly.

\vspace{2mm}

Moving from B to A, what is the price elasticity of demand? Again, show each step.

\vspace{5mm}

\textbf{Answer:}

\vspace{2mm}

Recall that $\Delta$ just means "change", and that when working with elasticities we don't really need to worry about negative and positive signs (this is clear from context).

\vspace{2mm}

Moving from A to B, the $\% \Delta$ in quantity demanded is:

$$\% \Delta Q_D = \dfrac{12-6}{6} = \dfrac{6}{6} = 1$$

Where we ignore the $\times 100 \%$, since it will drop out in the elasticity formula. The $\% \Delta$ in price is:

$$\% \Delta P = \dfrac{8-12}{12} = -\frac{1}{3} $$

So the price elasticity of demand is:

$$ \dfrac{1}{-1/3} = |-3| -= 3 $$

Now moving from B to A, the $\% \Delta$ in quantity demanded is:

$$\% \Delta Q_D = \dfrac{6-12}{12} = -\frac{1}{2}$$

The $\% \Delta$ in price is:

$$\% \Delta P = \dfrac{12-8}{8} = \dfrac{4}{8} = \dfrac{1}{2} $$

So the price elasticity of demand is:

$$ \frac{-1/2}{1/2} = |-1| = 1 $$ 

So we get two different numbers for elasticity, depending on whether we move from A to B or B to A. What gives?

\vspace{2mm}

This follows from a simple rule about percentages, but one that is easy to forget. If we want to change $X$ by $m\%$, we do so by multiplying $X \times (1 + \frac{m}{100})$. Letting $M = 1 + \frac{m}{100}$, we know that:

$$ X \neq (X \times (1 - \frac{m}{100})) \ times (1 + \frac{m}{100}) $$

This is probably best understood with an example. Say you have \$100. It might seem intuitive that if you lose 10\% and then immediately gain 10\%, you still have \$100. But this is not the case! If you have \$100, and pay taxes of 10\%, you now have \$90; if the next day the bank pays you a dividend of 10\%, you now only have \$99. This is the same idea that is at work in our case of differing elasticities depending on the our starting point.

\medskip


\item Using the mid-point formula, answer the following questions:
\begin{enumerate}
    \item What is the new base price?
    \item What is the new base quantity?
    \item What is the \% change for quantity?
    \item What is the \% change for price?
    \item What is the price elasticity of demand? Does it matter which point we treat as the start?
\end{enumerate}
\end{enumerate}

\vspace{2mm}

\textbf{Answer:}

\vspace{2mm}

\textbf{What is the new base price?}

\vspace{2mm}

$$\bar{P} = \dfrac{12 + 8}{2} = 10$$

\textbf{What is the new base quantity?}

$$\bar{Q} = \dfrac{6 + 12}{2} = 9$$

\textbf{What is the \% change for quantity?}

$$ \% \Delta Q = \dfrac{6-12}{9} = \dfrac{2}{3} $$

\textbf{What is the \% change for price?}

$$ \% \Delta P = \dfrac{8-12}{10} = \dfrac{-4}{10} = -\dfrac{2}{5}$$

\textbf{What is the price elasticity of demand? Does it matter which point we treat as the start?}

$$\text{Price elasticity of demand} = \dfrac{2/3}{2/5} = \dfrac{5}{3} $$

Unlike before, using the midpoint formula means that our starting points don't matter; check this for yourself by moving in the opposite direction as we did above, and verifying that the answer is unchanged.


\section*{Question 3}
Draw example demand curves which are:
\begin{itemize}
 \item Elastic
 \item Inelastic
 \item Unit elastic
 \item Perfectly elastic
 \item Perfectly inelastic
\end{itemize}
and provide the intuition behind the shape of each.

\vspace{2mm}

\textbf{Answer:}

\vspace{2mm}

Refer to figure 1 from the textbook (page 93) for examples. The intuition is:
\begin{itemize}
    \item Elastic: price change of $X\%$ $\to$ demand change greater than $X\%$ 
    \item Inelastic: price change of $X\%$ $\to$ demand change less than $X\%$ 
    \item Unit elastic: price change of $X\%$ $\to$ demand change of $X\%$
    \item Perfectly elastic: price change has no impact on demand
    \item Perfectly inelastic: small price change has enormous impact on demand
   \end{itemize}

\vspace{2mm}


\section*{Question 4}
Say price for some good doubles from $P_A$ to $P_B = 2*P_A$. How does total revenue change when:
    \begin{itemize}
        \item Demand is elastic: quantity decreases by 75\%
        \item Demand is inelastic: quantity decreases by 25\%
        \item Demand is unit elastic
    \end{itemize}

    \textbf{Answer:}

\vspace{2mm}

\textbf{Demand is elastic: quantity decreases by 75\%}

\vspace{2mm}

Since quantity decreases by 75\%, we can say:

$$Q_B = \dfrac{1}{4} Q_A$$.

Denote $TR_A$ as the total revenue before, where $TR_A = P_A \times Q_A$. Subbing in our new price and quantity formulas, we can see:

$$ TR_B = P_B \times Q_B = (2P_A) \times (\dfrac{1}{4} Q_A) = \dfrac{1}{2} P_A Q_A $$

So total revenue has decreased by $\frac{1}{2}$.

\vspace{2mm}

\textbf{Demand is inelastic: quantity decreases by 25\%}

Since quantity decreases by 25\%, we can say:

$$Q_B = \dfrac{3}{4} Q_A$$.

Subbing in our new price and quantity formulas:

$$ TR_B = P_B \times Q_B = (2P_A) \times (\dfrac{3}{4} Q_A) = \dfrac{6}{4} P_A Q_A = 1.5 \times P_A Q_A $$

So total revenue has now \textit{increased} by $\frac{1}{2}$.

\vspace{2mm}

\textbf{Demand is unit elastic}

$$ TR_B = 2 P_A (\frac{1}{2}Q_A) = P_A Q_A + TR_A $$

So total revenue is not changed.

\vspace{2mm}

\section*{Question 5}
Say we have a linear demand curve:
\begin{itemize}
    \item Quantity demanded is 0 when price is 100
    \item Quantity demanded is 10 when price is 20
\end{itemize}

\medskip

\begin{enumerate}
    \item Calculate the formula for the demand curve (slope and intercept) and draw graphically
    \item Is the elasticity constant? Why or why not?
    \item Pick a few example points, and use the midpoint formula to check the elasticity when:
        \begin{enumerate}
            \item Price is close to 20
            \item Price is close to 0
            \item Price is around 8
        \end{enumerate}
    \item How will total revenue vary as price moves from 0 to 100?
\end{enumerate}

\textbf{Answer:}

We are given two points, so we know we can find the slope:

$$ m = \dfrac{12-0}{4-100} = \dfrac{12}{-96} = -\dfrac{1}{8} $$

So we have $P = -\dfrac{1}{8}Q + b$ for some intercept $b$. To find the x-intercept, just plug in our point (0,100) and see that $b = 12.5$. So our demand curve is given by:

\begin{equation}
    \label{eq:demand}
    P = -\dfrac{1}{8} Q + 12.5
\end{equation}

We know that the elasticity will not be constant along a straight line, becaues of the rules of percentage changes. You could have picked any number of points: to find them, just choose price and use equation 1 to find the quantity. The prices I use are:
\begin{itemize}
    \item Point A: (0,100)
    \item Point B: (1,92)
    \item Point C: (6,52)
    \item Point D: (7,44)
    \item Point E: (12,4)
    \item Point F: (12.5,0)
\end{itemize}

We use the midpoint formula so that the direction of movement does not matter. 

\vspace{2mm}

A to B:

$$ \dfrac{(1-0)/0.5}{(92-100)/96} = 24 $$

\vspace{2mm}

C to D:

$$ \dfrac{(7-6)/6.5}{(44-52)/48} = \dfrac{14}{13} $$

\vspace{2mm}

E to F:

$$ \dfrac{(12.5-12)/12.25}{(0-4)/2} = \dfrac{1}{49} $$

As we would expect, near the y-axis when demand is almost 0, demand is very elastic (far above 1); when near the middle of the curve, demand is near unit elastic (around 1); when near the x-axis, when price is almost 0, demand is very inelastic (near 0).


\section*{Question 6}
Let's think about the market for hotel rooms, where we have some people searching for rooms for business travel, others for vacation, and some firms providing hotel rooms:

\begin{center}
\begin{table}[H]
  \centering
    \begin{tabular}{|c|c|c|c|}
      \hline
      \textbf{Price} & \textbf{$Q_D$ (Business)} & \textbf{$Q_D$ (Vacation)} & \textbf{$Q_S$ (Firms)} \\
      \hline
      \$150 & 2,100 & 1,000 & 2,300 \\
      \hline
      \$300 & 1,800 & 400   & 2,600 \\
      \hline
    \end{tabular}
    \caption{Market for hotel rooms}
  \end{table}
\end{center}

Which group do you expect to be elastic? Inelastic? Why?

\medskip
\medskip

Calculate the elasticities (using the midpoint method), and say when the market is inelastic.

\vspace{2mm}
\textbf{Answer:}

\vspace{2mm}

The elasticities are shown in table 2.

\begin{table}[H]
  \centering
    \begin{tabular}{|c|c|c|c|}
      \hline
      \textbf{Price} & \textbf{$Q_D$ (Business)} & \textbf{$Q_D$ (Vacation)} & \textbf{$Q_S$ (Firms)} \\
      \hline
      \$150 & 2,100 & 1,000 & 2,300 \\
      \hline
      \$300 & 1,800 & 400   & 2,600 \\
      \hline
    \end{tabular}
    \caption{Market for hotel rooms}
  \end{table}

Business demand is inelastic, while vacation demand is elastic. This is because business travelers have to go to a specific location at a specific time, and there are not many alternatives to getting a hotel room (forget aobut air bnb). Vacation travelers, on the other hand, may have a fair amount of flexibility in both \textit{where} and \textit{when} they go. An especially good answer will highlight how these things relate to the definition of the market we are looking at.

\vspace{2mm}

The firms providing hotel rooms are also fairly inelastic. This is because in the short-run the amount of hotel rooms is more-or-less fixed; it takes a lot of time and investment to build new capacity.

\vspace{2mm}

The market is in equilibrium when the price is \$250, because this is when the quantity supplied and demanded are equal to each other.


\end{document}
