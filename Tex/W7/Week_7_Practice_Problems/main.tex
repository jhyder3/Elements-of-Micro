\documentclass[12pt]{article}

%%%%%%%%%%%%%%%%%%%%%%%%%%%%
%%%%%%%%%%%%%%%%%%%%%%%%%%%%
% Load in packages
\usepackage{amsmath}
\usepackage{float}
\usepackage{amssymb}
\usepackage{hyperref}
\usepackage{graphicx}
\usepackage{enumitem}
\usepackage[top=1in, bottom=1in, left=1in, right=1in]{geometry}

%%%%%%%%%%%%%%%%%%%%%%%%%%%%
%%%%%%%%%%%%%%%%%%%%%%%%%%%%

\begin{document}

\begin{center}
\Large Chapter 8 Practice Problems

\medskip

\normalsize Elements of Microeconomics (discussion section 4)

\medskip

\small Jamie Hyder
\end{center}

\section*{Question 1}
Consider the market for coffee. The equations for quantity demanded and quantity supplied are as follows:
\begin{align*}
    Q_D &= 100-4P\\
    Q_S &= 5P + 10
\end{align*}
\begin{enumerate}[label=(\alph*)]

    \item Derive the inverse supply and demand equations ($P = ...$)
    \item Graph the inverse supply and demand equations.
    \item Is supply/demand elastic or inelastic in this case?
    \item Solve for the equilibrium price and quantity in this market, and illustrate this equilibrium on your graph.
    \item Indicate where on the graph represents the consumer and producer surplus in this market.
    \item Now, assume that there is a tax levied (on sellers) in the market for coffee in the amount of \$2 per cup. Does the supply or demand curve shift?
    \item What is the new equation for $Q_S$ \textit{and} the inverse supply curve?
    \item Add this new inverse supply curve to your existing graph. 
    \item Has the supply curve shifted left or right? By how much has the curve shifted?
    \item What is the new price faced by buyers in this market with the tax? What is the new price faced by sellers in this market with the tax?
    \item Indicate the areas on the graph which represent the producer surplus, consumer surplus, tax revenue, and dead weight loss.
    \item Calculate the values of the producer surplus, consumer surplus, tax revenue, and dead weight loss after the tax is levied in this market.
    \item Would the dead weight loss increase or decrease if the same tax was levied and (ceteris paribus) $Q_S = \frac{1}{2}P + 10$? What if (ceteris paribus) $Q_D = 100 - P$? Why is this the case? 
\end{enumerate}

\section*{Question 2}
Assume we are in the market for cars, and a tax has been levied on the car manufacturers of size $\$X$. Illustrate the dead weight loss caused by this tax in the following 4 scenarios on a supply/demand graph:
\begin{enumerate}[label = (\alph*)]
    \item Supply is elastic, demand is inelastic
    \item Supply is elastic, demand is elastic
    \item Demand is elastic, supply is inelastic
    \item Demand is elastic, supply is inelastic
\end{enumerate}


\section*{Question 3}
Consider the market for twizzlers. Assume demand and supply are both relatively elastic in this market. Illustrate the dead weight loss and tax revenue caused by a tax of the following three sizes:
\begin{enumerate}
    \item \(\$2\) per twizzler
    \item \(\$4\) per twizzler
    \item \(\$6\) per twizzler
\end{enumerate}

What happens to the size of the tax revenue as the size of the tax increases?

\vspace{3mm}

***(Note that I did not give any specific equations for $Q_S$ or $Q_D$, so you can't calculate the DWL or TR, I just want you to show what happens to DWL/TR as the tax per twizzler increases, nothing precise is necessary)

    


\end{document}
