\documentclass[12pt]{article}

%%%%%%%%%%%%%%%%%%%%%%%%%%%%
%%%%%%%%%%%%%%%%%%%%%%%%%%%%
% Load in packages
\usepackage{amsmath}
\usepackage{amssymb}
\usepackage{hyperref}
\usepackage{graphicx}
\usepackage[top=1in, bottom=1in, left=1in, right=1in]{geometry}

%%%%%%%%%%%%%%%%%%%%%%%%%%%%
%%%%%%%%%%%%%%%%%%%%%%%%%%%%

\begin{document}

\begin{center}
\Large Chapter 3 Practice Problems

\medskip

\normalsize Elements of Microeconomics (discussion section 4)

\medskip

\small Jamie Hyder
\end{center}

\medskip

\section*{Question 1}

Muffin's Steakhouse and Sandy's Salads are two restaurants which serve salads and steaks, respectively. Given 1000 minutes of labor time, they can produce the following amounts of each dish:

\begin{table}[h]
  \centering
  \begin{tabular}{|c|c|c|}
    \hline
    \textbf{Restaurant} & \textbf{Steaks} & \textbf{Salads} \\
    \hline
    \textbf{Muffin's Steakhouse} & 100 & 20 \\
    \hline
    \textbf{Sandy's Salads} & 200 & 100 \\
    \hline
  \end{tabular}
  \caption{Muffin vs. Sandy}
\end{table}

\begin{enumerate}

\item What is their cost, in minutes, to produce steak and salads?


\item Assume that there is a \textit{constant transferability} of productive resources from one dish to the other:
\begin{enumerate}
    \item Draw the production possibility frontiers for the two restaurants.
    \item Who has the absolute advantage in producing steaks?
    \item Who has the absolute advantage in producing salads?
\end{enumerate}


\item Let's think about the opportunity cost of each firm for each dish:
\begin{enumerate}
    \item What are the slopes of the two PPFs?
    \item What is Muffin's opportunity cost for producing steaks and salads?
    \item What is Sandy's opportunity cost for producing steaks and salads?
\end{enumerate}


\item 
\begin{enumerate}
    \item Can a firm have an absolute advantage in both goods?
    \item Can a firm have a comparative advantage in both goods?
    \item What is the relationship between the comparative advantage in good A and good B?
\end{enumerate}

\item Since most customers like to order a salad with their steak, Sandy and Muffin both want to offer both salads and steaks (not necessarily in equal quantities since some customers will only want one or the other).

\begin{enumerate}
    \item If both restaurants spend half their resources on each dish, what is their output?
    \item Assume the two businesses can trade. What is one set of productions, and one possible exchange, which would leave them both better off?
\end{enumerate}


\item We can think of the price of salads in terms of steaks in the above trade as 1 salad to 3 steaks. Would this trade still be profitable if:

\begin{enumerate}
    \item The price of 1 salad was 3.5 steaks?
    \item The price of 1 salad was 1 steak?
    \item The price of 1 salad was 6 steaks?
\end{enumerate}

\end{enumerate}

\section*{Question 2}
Joseph can peel a pound of potatoes in 10 minutes and wash a load of dishes in 15. Mary can do both of these tasks twice as fast. 

\medskip

Which person should do more of which task? (Don't worry about specific numbers.)

\section*{Question 3}
Joseph can peel a pound of potatoes in 10 minutes and wash a load of dishes in 15. Mary can also wash the dishes in 15 minutes, but it takes here only 5 minutes to peel the potatoes. 
    
\begin{enumerate}
\item 
    \begin{enumerate}
        \item What is each person's opportunity cost of peeling potatoes?
        \item Who has an absolute advantage in washing the dishes?
        \item Who has a comparative advantage in washing the dishes?
        \item If the two workers try and split up the tasks in an advantageous way, who will do more of which job?
    \end{enumerate}


\item Think about the price of peeling potatoes in terms of washing dishes. What is the maximum price at which a trade could leave both workers better off? What is the minimum price?

\medskip

\textbf{Bonus:} Now think about the price of washing dishes in terms of peeling potatoes. What is the range of possible prices? What is the relationship between the range of possible prices in this instance, vs. what we derived above?

\end{enumerate}

\end{document}
